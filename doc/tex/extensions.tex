\subsubsection{The Asterisk dialplan}

The Asterisk dialplan is divided into contexts. A context is simply a group
of extensions. For each "line" that should be able to be called, an extension
must be added to a context. Then, you configure the calling "line" to have
access to this context.

If you change the dialplan, you can use the Asterisk CLI command
"dialplan reload" to load the new dialplan without disrupting
service in your PBX.

Extensions are routed according to priority and may be based on any set
of characters (a-z), digits, \#, and *. Please note that when matching a
pattern, "N", "X", and "Z" are interpreted as classes of digits.

For each extension, several actions may be listed and must be given a unique
priority. When each action completes, the call continues at the next priority
(except for some modules which use explicitly GOTO's).

Extensions frequently have data they pass to the executing application
(most frequently a string).  You can see the available dialplan applications
by entering the "core show applications" command in the CLI.

In this version of Asterisk, dialplan functions are added. These can
be used as arguments to any application. For a list of the installed
functions in your Asterisk, use the "core show functions" command.

\subsubsection{Example dialplan}

The example dial plan, in the \path{configs/extensions.conf.sample} file
is installed as extensions.conf if you run "make samples" after
installation of Asterisk. This file includes many more instructions
and examples than this file, so it's worthwhile to read it.
	
\subsubsection{Special extensions}

There are some extensions with important meanings:

\begin{itemize}
  \item s	
  \begin{itemize}
    \item What to do when an extension context is entered (unless
          overridden by the low level channel interface)
          This is used in macros, and some special cases.
          "s" is not a generic catch-all wildcard extension.
  \end{itemize}
  \item i
  \begin{itemize}
    \item What to do if an invalid extension is entered
  \end{itemize}
  \item h
  \begin{itemize}
    \item The hangup extension, executed at hangup
  \end{itemize}
  \item t
  \begin{itemize}
    \item What to do if nothing is entered in the requisite amount
          of time.
  \end{itemize}
  \item T
  \begin{itemize}
    \item This is the extension that is executed when the 'absolute'
          timeout is reached.  See "core show function TIMEOUT" for more
          information on setting timeouts.	
  \end{itemize}
  \item e
  \begin{itemize}
    \item This extension will substitute as a catchall for any of the
          'i', 't', or 'T' extensions, if any of them do not exist and
          catching the error in a single routine is desired.  The
          function EXCEPTION may be used to query the type of exception
          or the location where it occurred.
  \end{itemize}
\end{itemize}

And finally, the extension context "default" is used when either a) an
extension context is deleted while an extension is in use, or b) a specific
starting extension handler has not been defined (unless overridden by the
low level channel interface).
