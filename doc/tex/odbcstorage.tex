

ODBC Storage allows you to store voicemail messages within a database 
instead of using a file.  This is \textbf{not} a full realtime engine and 
\textbf{only} supports ODBC.  The table description for the \texttt{voicemessages}
table is as follows:

\begin{table}[h]
\begin{center}
\begin{tabular}{ | l | l | c | c | l | l | }
\hline
Field          & Type        & Null & Key & Default & Extra \\ \hline \hline
msgnum         & int(11)     & Yes  &     & NULL    &       \\ \hline
dir            & varchar(80) & Yes  & MUL & NULL    &       \\ \hline
context        & varchar(80) & Yes  &     & NULL    &       \\ \hline
macrocontext   & varchar(80) & Yes  &     & NULL    &       \\ \hline
callerid       & varchar(40) & Yes  &     & NULL    &       \\ \hline
origtime       & varchar(40) & Yes  &     & NULL    &       \\ \hline
duration       & varchar(20) & Yes  &     & NULL    &       \\ \hline
flag           & varchar(8)  & Yes  &     & NULL    &       \\ \hline
mailboxuser    & varchar(80) & Yes  &     & NULL    &       \\ \hline
mailboxcontext & varchar(80) & Yes  &     & NULL    &       \\ \hline
recording      & longblob    & Yes  &     & NULL    &       \\
\hline
\end{tabular}
\end{center}
\caption{\texttt{voicemessages} Table Schema}
\end{table}

The database name (from \path{/etc/asterisk/res_odbc.conf}) is in the
\texttt{odbcstorage} variable in the general section of \path{voicemail.conf}.

You may modify the \texttt{voicemessages} table name by using
\texttt{odbctable=\textit{table\_name}} in \path{voicemail.conf}.
